% ŠABLONA PRO PSANÍ ZÁVĚREČNÉ STUDIJNÍ PRÁCE
%%%%%%%%%%%%%%%%%%%%%%%%%%%%%%%%%%%%%%%%%%%%
% Autor: Jakub Dokulil (kubadokulil99@gmail.com)
% Tato šablona byla vytvořena tak, aby pomocí ní mohli v systému LaTeX soutěžící sázet své práce a zároveň odpovídala požadavkům na formátování vyplývajícím z wordové šablony umístěné na webu soc.cz.
%
\documentclass[12pt, a4paper,
%oneside,      %% -- odkomentujte, pokud chcete svou práci mít pouze jednostrannou, mezera pro hřbet pak automaticky bude pouze na levé straně
twoside,        %% -- pro oboustranné práce, mezera pro hřbet následně střídá strany.
openright
]{report}

%% Nutné balíčky a nastavení
%%%%%%%%%%%%%%%%%%%%%%%%%%%%

%% Proměnné
\newcommand\obor{INFORMAČNÍ TECHNOLOGIE} %% -- napiš číslo a název tvého oboru
\newcommand\kodOboru{18-20-M/01} %% -- napiš číslo a název tvého oboru
\newcommand\zamereni{se zaměřením na počítačové sítě a programování} %% -- napiš číslo a název tvého oboru
\newcommand\skola{Střední škola průmyslová a umělecká, Opava} %% vyplň název školy
\newcommand\trida{IT4} %% vyplň jméno svého konzultanta
\newcommand\jmenoAutora{Mai Anh Perinová}  %% vyplň své jméno
\newcommand\skolniRok{2024/25} %% vyplň rok
\newcommand\datumOdevzdani{1. 1. 2024} %% vyplň rok
\newcommand\nazevPrace{HaruDolore} %% vyplň název své práce

\title{\nazevPrace} %% -- Název tvé práce
\author{\jmenoAutora} %% -- tvé jméno
\date{\datumOdevzdani} %% -- rok, kdy píšeš SOČku

\usepackage[top=2.5cm, bottom=2.5cm, left=3.5cm, right=1.5cm]{geometry} %% nastaví okraje, left -- vnitřní okraj, right -- vnější okraj

\usepackage[czech]{babel} %% balík babel pro sazbu v češtině
\usepackage[utf8]{inputenc} %% balíky pro kódování textu
\usepackage[T1]{fontenc}
\usepackage{cmap} %% balíček zajišťující, že vytvořené PDF bude prohledávatelné a kopírovatelné

\usepackage{graphicx} %% balík pro vkládání obrázků

\usepackage{subcaption} %% balíček pro vkládání podobrázků

\usepackage{hyperref} %% balíček, který v PDF vytváří odkazy

\linespread{1.25} %% řádkování
\setlength{\parskip}{0.5em} %% odsazení mezi odstavci


\usepackage[pagestyles]{titlesec} %% balíček pro úpravu stylu kapitol a sekcí
\titleformat{\chapter}[block]{\scshape\bfseries\LARGE}{\thechapter}{10pt}{\vspace{0pt}}[\vspace{-22pt}]
\titleformat{\section}[block]{\scshape\bfseries\Large}{\thesection}{10pt}{\vspace{0pt}}
\titleformat{\subsection}[block]{\bfseries\large}{\thesubsection}{10pt}{\vspace{0pt}}


\usepackage{tocloft} % Balíček umožní přizpůsobit vzhled tabulky obsahu
\setlength{\cftbeforechapskip}{0pt}  % Menší rozestup pro kapitoly
\setlength{\cftbeforesecskip}{0pt}   % Menší rozestup pro sekce

\setcounter{secnumdepth}{2}
\setcounter{tocdepth}{1}
\usepackage{fancyhdr}
\pagestyle{fancy}
\renewcommand{\headrulewidth}{0.025pt}

\usepackage{booktabs}

\usepackage{url}

%% Balíčky co se můžou hodit :) 
%%%%%%%%%%%%%%%%%%%%%%%%%%%%%%%

\usepackage{pdfpages} %% Balíček umožňující vkládat stránky z PDF souborů, 

\usepackage{upgreek} %% Balíček pro sazbu stojatých řeckých písmen, třeba u jednotky mikrometr. Například stojaté mí: \upmu, stojaté pí: \uppi

\usepackage{amsmath}    %% Balíčky amsmath a amsfonts 
\usepackage{amsfonts}   %% pro sazbu matematických symbolů
\usepackage{esint}     %% pro sazbu různých integrálů (např \oiint)
\usepackage{mathrsfs}
\usepackage{helvet} % Helvet font
\usepackage{mathptmx} % Times New Roman
\usepackage{Oswald} % Oswald font


%% makra pro sazbu matematiky
\newcommand{\dif}{\mathrm{d}} %% makro pro sazbu diferenciálu, místo toho
%% abych musel psát '\mathrm{d}' mi stačí napsat '\dif' což je mnohem 
%% kratší a mohu si tak usnadnit práci

\usepackage{listings}
\usepackage{xcolor}

\renewcommand{\lstlistingname}{Kód}% Listing -> Algorithm
\renewcommand{\lstlistlistingname}{Seznam programových kódů}% List of Listings -> List of Algorithms

%% Definice 
\lstdefinelanguage{JavaScript}{
	morekeywords=[1]{break, continue, delete, else, for, function, if, in,
		new, return, this, typeof, var, void, while, with},
	% Literals, primitive types, and reference types.
	morekeywords=[2]{false, null, true, boolean, number, undefined,
		Array, Boolean, Date, Math, Number, String, Object},
	% Built-ins.
	morekeywords=[3]{eval, parseInt, parseFloat, escape, unescape},
	sensitive,
	morecomment=[s]{/*}{*/},
	morecomment=[l]//,
	morecomment=[s]{/**}{*/}, % JavaDoc style comments
	morestring=[b]',
	morestring=[b]"
}[keywords, comments, strings]


\lstdefinelanguage[ECMAScript2015]{JavaScript}[]{JavaScript}{
	morekeywords=[1]{await, async, case, catch, class, const, default, do,
		enum, export, extends, finally, from, implements, import, instanceof,
		let, static, super, switch, throw, try},
	morestring=[b]` % Interpolation strings.
}

\lstalias[]{ES6}[ECMAScript2015]{JavaScript}

% Nastavení barev
% Requires package: color.
\definecolor{mediumgray}{rgb}{0.3, 0.4, 0.4}
\definecolor{mediumblue}{rgb}{0.0, 0.0, 0.8}
\definecolor{forestgreen}{rgb}{0.13, 0.55, 0.13}
\definecolor{darkviolet}{rgb}{0.58, 0.0, 0.83}
\definecolor{royalblue}{rgb}{0.25, 0.41, 0.88}
\definecolor{crimson}{rgb}{0.86, 0.8, 0.24}

% Nastavení pro Python
\lstdefinestyle{Python}{
	language=Python,
	backgroundcolor=\color{white},
	basicstyle=\ttfamily,
	breakatwhitespace=false,
	breaklines=false,
	captionpos=b,
	columns=fullflexible,
	commentstyle=\color{mediumgray}\upshape,
	emph={},
	emphstyle=\color{crimson},
	extendedchars=true,  % requires inputenc
	fontadjust=true,
	frame=single,
	identifierstyle=\color{black},
	keepspaces=true,
	keywordstyle=\color{mediumblue},
	keywordstyle={[2]\color{darkviolet}},
	keywordstyle={[3]\color{royalblue}},
	literate=%
	{á}{{\'a}}1 {č}{{\v{c}}}1 {ď}{{\v{d}}}1 {é}{{\'e}}1 {ě}{{\v{e}}}1
	{í}{{\'i}}1 {ň}{{\v{n}}}1 {ó}{{\'o}}1 {ř}{{\v{r}}}1 {š}{{\v{s}}}1
	{ť}{{\v{t}}}1 {ú}{{\'u}}1 {ů}{{\r{u}}}1 {ý}{{\'y}}1 {ž}{{\v{z}}}1,		
	numbers=left,
	numbersep=5pt,
	numberstyle=\tiny\color{black},
	rulecolor=\color{black},
	showlines=true,
	showspaces=false,
	showstringspaces=false,
	showtabs=false,
	stringstyle=\color{forestgreen},
	tabsize=2,
	title=\lstname,
	upquote=true  % requires textcomp	
}


\lstdefinestyle{JSES6Base}{
	backgroundcolor=\color{white},
	basicstyle=\ttfamily,
	breakatwhitespace=false,
	breaklines=false,
	captionpos=b,
	columns=fullflexible,
	commentstyle=\color{mediumgray}\upshape,
	emph={},
	emphstyle=\color{crimson},
	extendedchars=true,  % requires inputenc
	fontadjust=true,
	frame=single,
	identifierstyle=\color{black},
	keepspaces=true,
	keywordstyle=\color{mediumblue},
	keywordstyle={[2]\color{darkviolet}},
	keywordstyle={[3]\color{royalblue}},
 literate=%
{á}{{\'a}}1 {č}{{\v{c}}}1 {ď}{{\v{d}}}1 {é}{{\'e}}1 {ě}{{\v{e}}}1
{í}{{\'i}}1 {ň}{{\v{n}}}1 {ó}{{\'o}}1 {ř}{{\v{r}}}1 {š}{{\v{s}}}1
{ť}{{\v{t}}}1 {ú}{{\'u}}1 {ů}{{\r{u}}}1 {ý}{{\'y}}1 {ž}{{\v{z}}}1,		
	numbers=left,
	numbersep=5pt,
	numberstyle=\tiny\color{black},
	rulecolor=\color{black},
	showlines=true,
	showspaces=false,
	showstringspaces=false,
	showtabs=false,
	stringstyle=\color{forestgreen},
	tabsize=2,
	title=\lstname,
	upquote=true  % requires textcomp
}

\lstdefinestyle{JavaScript}{
	language=JavaScript,
	style=JSES6Base,
}
\lstdefinestyle{ES6}{
	language=ES6,
	style=JSES6Base
}


%% Bordel pro práci - můžeš smáznout :) 
%%%%%%%%%%%%%%%%%%%

\usepackage{lipsum} %% balíček který píše lipsum (nesmyslný text, který se používá pro kontrolu typografie)

%% Začátek dokumentu
%%%%%%%%%%%%%%%%%%%%
\begin{document}
	
	\pagestyle{empty}
	\pagenumbering{Roman}
	
	\cleardoublepage

%% Titulní stránka s informacemi
%%%%%%%%%%%%%%%%%%%%%%%%%%%%%%%%%%%%%%%%
	
	{\fontfamily{phv}\selectfont
		%% Logo školy
		\begin{figure}[h]
			\centering
			\includegraphics[width=0.6\linewidth]{image/logo-skoly.png} 
		\end{figure}
		
		
		%% Hlavička práce a její název (viz proměnná \nazev prace)
		%% \sffamily %%% bezpatkové písmo - sans serif
		{\bfseries %%% písmo na stránce je tučně
			\begin{center}
				\vspace{0.025 \textheight}
				\LARGE{ZÁVĚREČNÁ STUDIJNÍ PRÁCE}\\
				\large{dokumentace}\\
				\vspace{0.075 \textheight}
				\LARGE {\nazevPrace}\\
			\end{center}  
		}%%%
		
		\begin{figure}[h]
			\centering
			\includegraphics[width=0.8\linewidth]{image/programovani-02.jpg} 
		\end{figure}
		
		\vspace{0.02 \textheight}
		\begin{table}[h!]
			\begin{tabular}{ll}
				\textbf{Autor:} & \jmenoAutora\\ 
				\textbf{Obor:} & \kodOboru { } \obor\\
				\textbf{} & \zamereni\\
				\textbf{Třída:} & \trida\\
				\textbf{Školní rok:} & \skolniRok\\
			\end{tabular}
			
		\end{table}		
	}
	
\clearpage %% Zalomení stránky
	
%% Stránka obsahující poděkování a prohlášení
%%%%%%%%%%%%%%%%%%%%%%%%%%%%%%%%%%%%%%%%%%%%%%%%%%%%%%%%
	
	\vspace*{0.7\textheight} %% Vertikální mezeru je možné upravit

%% Prohlášení - povinné
%%%%%%%%%%%%%%%%%%%%%%%%%%%%
	\noindent{\large{\bfseries{Prohlášení}\\}}  %% uprav si koncovky podle toho na jaký rod se cítíš, vypadá to pak lépe :) 
	\noindent{Prohlašuji, že jsem závěrečnou práci vypracoval samostatně a uvedl veškeré použité 
		informační zdroje.\\}
	\noindent{Souhlasím, aby tato studijní práce byla použita k výukovým a prezentačním účelům na Střední průmyslové a umělecké škole v Opavě, Praskova 399/8.}
	\vfill
	\noindent{V Opavě \datumOdevzdani\\}
	\noindent
	\begin{minipage}{\linewidth}
		\hspace{9.5cm} 
		\begin{tabular}{@{}p{6cm}@{}}
			\dotfill \\
			Podpis autora
		\end{tabular}
	\end{minipage}
	
	\clearpage %% Zalomení stránky

%% Stránka obsahující abstrakt (anotaci)
%%%%%%%%%%%%%%%%%%%%%%%%%%%%%%%%%%%%%%%%%%%%%%%%%%%%%%%%	

%% Abstrakt v češtině
%%%%%%%%%%%%%%%%%%%%%%%%%%%%
	\noindent{\Large{\bfseries{Abstrakt}\\}}
	Výsledkem projektu je funkční webová aplikace pro ukládání souborů k určité maturitní otázce. Aplikace umožňuje přihlášení uživatele přes Email, Google, GitHub a Microsoft. Stránka obsahuje maturitní otázky, které jsou rozděleny do určitých kategorií a jsou přehledně zobrazeny. Uživatel si vybírá otázku ke které pak přidá výukový soubor. Soubory se ukládají ke každé otázce zvlášť a jsou k dispozici ke stáhnutí pro přihlášeného uživatele. Při nahrávání výukového matetirálu musí uživatel soubor zařadit do kategorie, která se pak zobrazuje jako štítek vedle nahraného souboru.
	\\
	
	\vspace{18pt}
	
	\noindent{\large{\bfseries{Klíčová slova}}}
	
	\noindent webová stránka, databáze, uživatelské účty, soubory \dots 
	
	\vspace{18pt}
	
	\clearpage %% Zalomení stránky

%% Stránka s generovaným obsahem
%%%%%%%%%%%%%%%%%%%%%%%%%%%%%%%%%%%%%%%	
	
	\tableofcontents %% Vygeneruje tabulku s obsahem

	\pagenumbering{arabic} %% Nastavení způsobu číslování stránek (alternativy roman | Roman)
	\setcounter{page}{1} %% Nastavení počitadla stránek

%% Stránka s úvodem - povinná část
%%%%%%%%%%%%%%%%%%%%%%%%%%%%%%%%%%%%%%%		
	\chapter*{Úvod}
%Tento příkaz vytvoří novou kapitolu s názvem "Úvod" ve vašem dokumentu.
%Hvězdička * u příkazu \chapter* znamená, že tato kapitola nebude mít číslo. Ve výsledném dokumentu se tedy objeví jako "Úvod" bez předcházejícího čísla kapitoly, které se obvykle zobrazuje u číslovaných kapitol.
%Tento příkaz také znamená, že kapitola se automaticky neobjeví v obsahu, protože LaTeX standardně zahrnuje do obsahu pouze číslované kapitoly.
	\addcontentsline{toc}{chapter}{Úvod}
%Tento příkaz ručně přidává záznam do obsahu.
%První parametr toc označuje, že přidáváme záznam do Table of Contents (obsahu).
%Druhý parametr chapter specifikuje úroveň záznamu. V tomto případě říkáme, že přidávaný záznam má být považován za kapitolu.
%Třetí parametr Úvod je text, který se objeví v obsahu. V tomto případě bude v obsahu zobrazen název "Úvod".	
Závěrečné studijní práce a jejich veřejné obhajoby jsou důležitou součástí vyvrcholení studia oboru informační technologie na \textit{Střední škole průmyslové a umělecké v Opavě}. Hlavním cílem je samostatně vypracovat komplexní, nejčastěji prakticky zaměřený projekt na vybrané téma z~oblasti ICT a napsat k tomuto projektu také příslušnou odbornou dokumentaci podle obecně platných pravidel. Nejúspěšnější studentské projekty bývají vybrány, aby školu reprezentovaly v soutěžních přehlídkách \emph{Středoškolské odborné činnosti} (dále SOČ), kde je pečlivá a přesná dokumentace rovněž vyžadována.

Tento dokument vznikl se záměrem co nejvíce studentům usnadnit formální úpravu dokumentace k jejich odborné práci a poskytnout jim i dobré vodítko při strukturování samotného obsahu. Vzhledem k tomu, že příprava na vysokoškolské studium často vyžaduje znalost {\LaTeX}u, představujeme šablonu vytvořenou v této technologii, jejímž autorem je z velké části Jakub Dokulil a která byla původně určená pro soutěžící SOČ.  

První kapitola této práce obsahuje základní informace o~{\LaTeX}u; stručně se zmiňuje o jeho vývoji, ale hlavně se soustředí na základní principy, které je nezbytné znát při sestavování rozsáhlejších  textových dokumentů. Součástí této kapitoly je i výběr některých programových prostředků, které mohou výrazněji urychlit a usnadnit psaní zdrojového kódu {\LaTeX}u. V kapitole nazvané "Jak psát odbornou dokumentaci" jsou zdůrazněna nejdůležitější pravidla, jež by měla být dodržena při psaní (nejen) odborného textu, a to jak zásady týkající se obsahu, tak i formální stránky - pravopisné, typografické apod. Závěrečná kapitola se soustředí na praktické ukázky správného použití různých typů obsahu v odborné práci. 

%Tipy k psaní úvodu
%Je povinný, nadpis neměňte, rozsah - max. 1 strana. 
%Tato část práce obsahuje: 
%* náhled do řešené problematiky, zdůvodnění volby problematiky, 
%* předem definované cíle práce, 
%* motivaci pro další čtení textu včetně stručného uvedení obsahu následujících kapitol 


\chapter{Struktura aplikace}
V této kapitole si popíšeme strukturu webové aplikace.
\section{Frontend}
\label{sec:frontend}


\subsection{React Next.js}


\subsection{Tailwind CSS}
Hlavní výhodou používání \LaTeX{} je jeho schopnost vytvářet profesionálně vypadající dokumenty s konzistentním formátováním. Dále nabízí:

\begin{itemize}
	\item Vynikající kvalitu sazby, zvláště pro matematické vzorce.
	\item Automatizované generování obsahu, seznamů obrázků, tabulek a bibliografických odkazů.
	\item Možnost snadno pracovat s komplexními dokumenty jako jsou disertace nebo knihy.
	\item Rozsáhlé možnosti přizpůsobení a širokou škálu balíčků rozšiřujících jeho funkčnost.
\end{itemize}

V následujících sekcích se podrobněji podíváme na základní prvky \LaTeX{} a naučíme se, jak je používat k vytváření kvalitních dokumentů.

\subsection{Clerk}
Hlavní výhodou používání \LaTeX{} je jeho schopnost vytvářet profesionálně vypadající dokumenty s konzistentním formátováním. Dále nabízí:


\section{Backend}
\label{sec:backend}

V této kapitole se podrobněji podíváme na základní strukturu dokumentu v \LaTeX{}u. Po porozumění této struktuře budete schopni vytvářet vlastní dokumenty s přizpůsobeným formátováním a strukturou.

\subsection{JavaScript}
Preambule je první částí každého \LaTeX{}ového dokumentu. Zde definujeme typ dokumentu, který chceme vytvořit, a nastavíme různé parametry, které ovlivňují celkový vzhled dokumentu. Preambule také často obsahuje příkazy pro načítání různých balíčků, které rozšiřují základní funkčnost \LaTeX{}u.

\begin{verbatim}
	\documentclass[options]{class}
	\usepackage[options]{package}
\end{verbatim}

\subsection{Strapi GraphQl}
Hlavní tělo dokumentu začíná příkazem \verb|\begin{document}| a končí \verb|\end{document}|. Veškerý obsah, který chcete mít ve svém dokumentu, by měl být umístěn mezi tyto dva příkazy. 

\subsection{Model databáze}
Pro organizaci obsahu se často používají sekce a podsekce. Tyto struktury pomáhají čtenáři lépe navigovat dokumentem a rozdělit text do logických bloků.

	\chapter{Způsoby řešení a použité postupy}
	\pagestyle{fancy}
	Jak už jsem psal výše \LaTeX je dosti komplexní systém, který umožňuje psát velmi rozsáhlé text. Jeho autor Donald Knuth ho stvořil, aby mohl vydat jeho učebnici \emph{The Art of Computer Programming} a dodnes se je využíván pro sazbu skript, učebnic, článků či závěrečných prací. V této kapitole najdeš ukázky různých funkcí a balíčků \LaTeX u od těch nejzákladnějších až po složitější. Neznamená to nutně, že všechny musíš použít, ale když potřebuješ pomoct, tak je dobré mít oporu. 
	
	Pokud s \LaTeX em úplně začínáš tak ti můžu doporučit přiručku \emph{Ne příliš stručný úvod do systému \LaTeX2e}~\cite{LaTeXprirucka}. Případně spoustu užitečných informací nalezneš na Wikibooks~\cite{wikibooksLaTeX}. Pokud narazíš na nějaký problém googli. Na internetu je spoustu fór, kde pravděpodobně už někdo podobný problém řešil. Asi nejvíce otho najdeš na stránce \emph{TeX - LaTeX Stackexchange} \cite{stackExchange}.
	
	
	\section{Založení projektu} %%[Text, který bude v obsahu]{Text, který se vytiskne na stránce} Zkus měnit jednotlivé závorky a uvidíš :) 
	Psaní v \LaTeX{u} není žádná věda, stačí psát normálně do zdrojového souboru. Pokud bys chtěl psát obrážky či číslovaný seznam, pak můžeš použít prostředí \texttt{itemize} či \texttt{enumerate}. Často je důležité používat nezlomitelnou mezeru. Tu uděláš pomocí \verb|~|~(tildy). Pokud budeš chtít psát uvozovky použij příkaz \texttt{uv}, pomocí něj se ti vytvoří uvozovky podle příslušného jazyka. V česku tedy ve formátu 99 66. Použití příkazu najdeš níže v textu.
	
	Občas je zapotřebí \LaTeX{u} pomoct při rozdělování slov. To se udělá snadno vložením symbolů \verb|\-| mezi jednotlivé slabiky.
	
\begin{lstlisting}[style=Python, caption={Ukázka Python kódu}]
	# Python code here
	def hello_world():
		print("Hello, world!")
\end{lstlisting}


\begin{lstlisting}[style=JavaScript, title={Kód}, caption={Ukázka JS kódu}]
	// JavaScript code here
	function helloWorld() {
		console.log("Hello, world!");
	}
\end{lstlisting}	
	
\begin{lstlisting}[style=ES6, caption={ES6 (ECMAScript-2015) Listing}]
	/* eslint-env es6 */
	/* eslint-disable no-unused-vars */
	
	import Axios from 'axios'
	import { BASE_URL } from './utils/api'
	import { getAPIToken } from './utils/helpers'
	
	export default class User {
		constructor () {
			this.id = null
			this.username = null
			this.email = ''
			this.isActive = false
			this.lastLogin = ''  // ISO 8601 formatted timestamp.
			this.lastPWChange = ''  // ISO 8601 formatted timestamp.
		}
	}
	
	const getUserProfile = async (id) => {
		let user = new User()
		await Axios.get(
		`${BASE_URL}/users/${id}`,
		{
			headers: {
				'Authorization': `Token ${getAPIToken()}`,
			}
		}
		).then{response => {
				// ...
			}).catch(error => {
				// ...
			})
		}
\end{lstlisting}	
	
	\section{Autentizace}
	
	\subsection{Základní operace s uživatelským účtem}
	
	Sazba matematiky je věda sama o sobě. Ačkoli Word prošel obrovskou změnou a je v~tomto mnohem lepší, tak \LaTeX je pro to přímo (ještě jsem neviděl matematika, co by používal Word). Spolu s balíčky \texttt{amsmath} a \texttt{amsfonts} snad neexistuje nic, co by se používalo a \LaTeX by to nezvládl. Ať už jde o základní věci jako řecká písmenka -- $\alpha, \beta, \gamma, \dots$ -- integrály -- $\int_{l_i}^{l_f} \tau \dif l $ -- až třeba po speciální písmena -- $\mathscr{F}: \mathbb{R}^n \to \mathbb{R}^m$. Pro případ, že bys potřeboval nějaké speciální integrály, je tu balíček \texttt{esint}, pomocí něj můžeš napsat třeba
	$$ \oiint_{S(V)} \vec{E} \cdot \dif \vec{S} = \iiint_{V} \left(\vec{\nabla} \cdot \vec{E}\right) \dif V .$$
	
	Jak můžeš vidět tak rovnice lze psát jednak do textu a nebo pokud se jedná o nějakou důležitou nebo rozsáhlejší rovnici tak na samostatný řádek. Pokud je rovnice opravdu důležitá, tak je vhodné ji také číslovat. Pak se na ni můžeš dále odkazovat v textu.
	\begin{equation}
		\vec{F} = m \vec{a}
		\label{eq:newton2}
	\end{equation}
	\dots Například podle druhého Newtonova zákona, rovnice (\ref{eq:newton2}) \dots Zároveň je vždy nutné vysvětlit co která veličina znamená. V tomto případě bych napsal, že v druhém Newtonově zákoně vektor síly $\vec F$ odpovídá součinu hmotnosti tělesa $m$ a jeho zrychlení $\vec a$. 
	
	Věřím, že se sazbou matematiky ti pomůže tvůj školitel, případně mi můžeš napsat (mail je v úvodu). Jednotlivé funkcionality spolu se seznamem znaků nalezneš jednak v Ne příliš stručném úvodu~\cite{LaTeXprirucka} nebo na Wikibooks v sekcích \emph{Mathematics} a \emph{Advanced mathematics}~\cite{wikibooksLaTeX}.
	
	
	
	\chapter{Funkce aplikace}
	
	Každou práci je dobré zkontrolovat, aby v ní nebyly pravopisné chyby, nebyla těžkopádně napsaná -- byla čtivá -- a neobsahovala žádný typografický nedostatek. Proto, když práci sepíšeš, nech ji chvilku odležet, třeba týden. Pak si ji po sobě znovu přečti. Hned uvidíš, kolik věcí bys napsal jinak případně kde tě bije do očí jaká chyba. Dej práci přečíst také svému školiteli a případně češtináři. Zajistíš tak, že bude obsahovat méně chyb.
	
	Pak můžeš práci vytisknout a hurá do soutěže.
	
	\chapter*{Závěr}
	
	Věřím, že jsem ti spolu se šablonou poskytl několik tipů, jak napsat práci. Ať už jde o úplné začátky s \LaTeX{}em. Či ukázku toho, co vše s ním zvládneš. Pokud bys měl k šabloně libovolné dotazy, rouhodně se na mě obrať. \LaTeX tvé práci dodá určitou krásu, tak doufám, že ti dodá sebevědomí a uspěješ při souteži. A i kdyby ne vzpomeň si, kolik ses toho musel naučit a hned uvidíš o jaký kus ses posunul.
	
	%% literatura
	\begin{thebibliography}{99}
		\bibitem{sablonaSOC} DOKULIL Jakub. \textit{Šablona pro psaní SOČ v programu \LaTeX} [Online]. Brno, 2020 [cit. 2020-08-24]. Dostupné z: \url{https://github.com/Kubiczek36/SOC_sablona}
		\bibitem{LaTeXprirucka}OETIKER, Tobias, Hubert PARTL, Irene HYNA, Elisabeth SCHEGL, Michal KOČER a Pavel SÝKORA. \textit{Ne příliš stručný úvod do systému LaTeX2e} [online]. 1998 [cit. 2020-08-24]. Dostupné z: \url{https://www.jaroska.cz/elearning/informatika/typografie/lshort2e-cz.pdf}
		\bibitem{wikibooksLaTeX}\textit{Wikibooks: LaTeX} [online]. San Francisco (CA): Wikimedia Foundation, 2001- [cit. 2020-08-24]. Dostupné z: \url{https://en.wikibooks.org/wiki/LaTeX}
		\bibitem{stackExchange} \textit{TeX - LaTeX Stack Exchange} [online]. Stack Exchange, 2020 [cit. 2020-09-01]. Dostupné z: \url{https://tex.stackexchange.com}
		\bibitem{sspuLogo} \textit{Střední škola průmyslová a umělecká Opava} [online]. [cit. 2023-11-11]. Dostupné z: \url{https://www.sspu-opava.cz}
		\bibitem{citacePRO}\textit{Citace PRO} [online]. Citace.com, 2020 [cit. 2020-08-31]. Dostupné z: \url{https://www.citacepro.com}
		\bibitem{Born2019} BORN, Max a Emil WOLF. \textit{Principles of optics: electromagnetic theory of propagation, interference and diffraction of light}. 7th (expanded) edition. Reprinted wirth corrections 2002. 15th printing 2019. Cambridge: Cambridge University Press, 2019. ISBN 978-0-521-64222-4.
	\end{thebibliography}
	
	%% obrázky 
	\listoffigures
	
	%% tabulky
	\listoftables
	
	\appendix %% začínají přílohy
	
	\titleformat{\chapter}[block]{\scshape\bfseries\LARGE}{Příloha \thechapter}{10pt}{\vspace{0pt}}[\vspace{-22pt}] %% nastavení nadpisu u příloh
	
	
\end{document}